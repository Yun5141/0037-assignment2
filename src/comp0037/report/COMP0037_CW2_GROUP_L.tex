\documentclass{article}


\usepackage{arxiv}

\usepackage{placeins}
\usepackage[utf8]{inputenc} % allow utf-8 input
\usepackage[T1]{fontenc}    % use 8-bit T1 fonts
\usepackage{hyperref}       % hyperlinks
\usepackage{url}            % simple URL typesetting
\usepackage{booktabs}       % professional-quality tables
\usepackage{amsfonts}       % blackboard math symbols
\usepackage{nicefrac}       % compact symbols for 1/2, etc.
\usepackage{microtype}      % microtypography

\usepackage{graphicx}	% to insert graphs
\usepackage{caption}	% to customize caption style
\usepackage{float}
\usepackage{subfigure}


\title{COMP0037 ASSIGNMENT 2}


\author{
 Group: \texttt{Group L}\\
}

\date{}

\begin{document}

\maketitle

\captionsetup[figure]{labelformat={default},labelsep=period,name={Fig.}}


% -------------------------------------------------------------------------------------------
\section{Introduction [dai ding]} 

% -------------------------------------------------------------------------------------------
\section{Reactive Planner [Yun]}

\subsection {Definition of Reactive Planning System}

\subsection {Implementation of Our Reactive Planning System}

\subsection {Approach for Improving the Performance}

% -------------------------------------------------------------------------------------------
\section{Frontier-Based Exploration System [Yun]}

\subsection {Frontiers}

\subsection {The Exploration Algorithm Provided}

\subsection {Our Implementation}

% -------------------------------------------------------------------------------------------
\section{Integration of Our Planner and Exploration System [Yun]}

% -------------------------------------------------------------------------------------------
\section{Information-Theoretic Path Planning [Yusi]}

% -------------------------------------------------------------------------------------------
\section{Conclusion [dai ding]}

% -------------------------------------------------------------------------------------------
\bibliographystyle{unsrt}  
%\bibliography{references}  %%% Remove comment to use the external .bib file (using bibtex).
%%% and comment out the ``thebibliography'' section.


%%% Comment out this section when you \bibliography{references} is enabled.
\begin{thebibliography}{1}

\end{thebibliography}


% -----------------------------------------------------------------------------------------
\end{document}